\chapter{Desenvolvimento do software}
	\label{ch:desenvolvimento}
Neste capítulo é discutido um pouco sobre o desenvolvimento do \textit{software} de tratamento de dados proposto por este trabalho. 

O programa feito para os boxes é implementado em sua totalidade na linguagem interpretada Python, esta foi escolhida devido a facilidades que a mesma proporciona no uso de bibliotecas de suporte ao sistema e de interfaceamento gráfico além de ser uma linguagem comum ao autor. Outras linguagens como o C++ foram consideras para a aplicação, porém como o fluxo de dados não é feito em grande escala a velocidade de processamento não é uma questão primária. O sistema que atua no microcontrolador é programado em C e C++, pois este é o padrão de desenvolvimento para os microcontroladores utilizados. 

\section{Bibliotecas utilizadas}
O sistema de tratamento de dados utiliza de diversos recursos disponíveis na linguagem Python, nesta seção será descrito quais são e o que fazem. 

Um dos principais componentes do \textit{software} é a interface gráfica, ou interface do usuário. É por meio dela que a equipe presente nos boxes pode ter acesso as informações do veículo, observando os gráficos de tração, velocidade, extensometria, etc... Para a concepção da interface gráfica foi utilizada uma \textit{framework} chamada Qt \footnote[1]{Site: \url{https://www.qt.io/}}. O Qt é nativamente utilizado para o C++ na construção de interfaces gráficas, para utilizar-lo no Python, é necessário a utilização de uma biblioteca auxiliar chamada PyQt\footnote[2]{Site: \url{https://riverbankcomputing.com/software/pyqt/intro}}. Tanto o Qt quanto o PyQt estão disponíveis em Windows, distribuições Linux e MacOS, um fator importante visto que a equipe utiliza o sistema operacional Windows, porém o autor desenvolveu o sistema utilizando a distribuição Manjaro XFCE e Ubuntu 16.04. Este é um tema recorrente neste trabalho, todas as bibliotecas utilizadas devem ser portáveis para ambos os sistemas operacionais Windows ou distribuições Linux como declarado nos objetivos específicos. As versões utilizadas das bibliotecas foram o Qt 5.5.1 e o PyQt 5

Para o desenvolvimento foi utilizado a ferramenta Designer da \textit{framework} Qt. A ferramenta permite a criação de telas com auxílio de interface gráfica e após a construção das mesmas telas, se utiliza um comando de linha (\textit{pyuic5}) para transformar o código bruto da tela de XML para Python. Este novo código em Python é utilizado para se referenciar os objetos dentro da interface, necessário para união da parte lógica com a parte gráfica.

Uma característica específica deste sistema é sua execução em tempo real. Isto implica que a parte gráfica deve rodar em conjunto com o recebimento dos dados enviados pelo veículo, tendo isto em mente, é necessário utilizar de processamento paralelo para fazer ambas as tarefas. A solução encontrada para tal foi a biblioteca \textit{multiprocessing} do Python. Esta biblioteca possui funções para criação de \textit{pipes}, estruturas que retorna dois objetos representando um "cano" de passagem de informação entre dois processos, estes dois processos tem acesso a comandos de envio e recepção de dados para conversa entre duas aplicações em tempo real. A função \textit{pipe} opera de forma \textit{duplex}, onde ambos os lados podem enviar e receber dados, porém devido ao escopo atual do projeto o sistema será utilizado como \textit{simplex}.    

Para a conexão do sistema de boxes com o microcontrolador é utilizado o protocolo de rede sem fio ZigBee. O protocolo é definido como uma solução para sistemas de baixa taxa de transmissão de dados e baixo consumo de energia, ele opera no alcance de uma PAN (\textit{Personal Area Network}) e tem uma taxa de transferência de até 250Kbps \cite{elahi2009zigbee}. Estas características levaram a utilização deste protocolo para a comunicação sem fio entre os sistemas citados.