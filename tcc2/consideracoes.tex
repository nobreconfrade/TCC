\chapter{Conclusão e Trabalhos Futuros}
	\label{ch:consideracoes}
O desenvolvimento de um \textit{software} de tratamento de dados para o veículo da categoria Baja SAE se mostrou necessário devido ao nível de complexidade e quantidade de sensores que fazem e devem fazer parte do projeto da equipe Velociraptor. Apenas a utilização de arquivos de texto para gerenciamento das informações pode deixar a tarefa confusa, árdua e dificulta o armazenamento das informações de forma compreensiva e centralizada. O sistema produzido alcança o objetivo geral estipulado, abrangendo todos os sensores contidos no veículo e alguns que ainda se encontram em desenvolvimento já estão implementados no sistema de tratamento. Ele também auxilia na manutenção indicando possíveis falhas para a equipe caso alguma grandeza saia dos padrões estabelecidos durante testes ou provas. Além da parte de \textit{software} possuir essas características, o sistema foi utilizado em um teste prático e com resultados reais, no qual a equipe pode verificar o funcionamento de algumas grandezas importantes do sistema, os dados da suspensão traseira e temperatura de operação do motor. Tendo isto dito, foi possível verificar que as características propostas no início do Capitulo \ref{ch:proposta} de apresentação dos dados de forma intuitiva, simplificação de complexidade com a junção dos dados em um único ambiente, modularidade para adição de novas tecnologias e armazenamento dos dados tratados foram atingidas devido aos testes realizados e também ao \textit{feedback} recebido dos membros da equipe Velociraptor.

% As barreiras de levantamento de requisitos existentes eram foram quebradas com a constante participação do cliente na concepção do projeto e no desenvolvimento do mesmo. O sistema que foi criado a partir de conversar e reuniões atendeu as expectativas iniciais da equipe, porém melhorias ainda devem ser feitas devido ao fato do objetivo final do projeto é ter o sistema utilizado nas competições do Baja Sul 2018 e Baja Nacional 2018. 

Em relação a trabalhos futuros o sistema deve passar por uma etapa de refinamento das funções e de pesquisa de novas funcionalidades. Na etapa de refinamento, é importante destacar que a taxa de atualização deve ser mantida em uma atualização a cada 500 ms quando adicionados novos sensores ao veículo, considerando sensores de uso em competição. Os sensores que não estavam prontos para os testes finais devem ser adicionados ao veículo para realização de um teste completo do sistema antes do inicio do Baja Sul 2018. Os sensores de extensometria serão implementados também ao sistema, logo que a parte de \textit{hardware} seja projetada e implementada pela equipe de eletrônica embarcada. Por último, novos requisitos de testes podem surgir e a criação de novos parâmetros para apresentação de gráficos podem se tornar necessários, um exemplo disto é a criação de uma janela específica para a realização de testes de aceleração do veículo.

Em relação a etapa de pesquisa para novas funcionalidade que podem ser pesquisadas e implementadas em trabalhos futuros, deve se tomar como diretriz o que está acontecendo no mundo automotivo atualmente. Um salto de complexidade para o sistema e de entrega de valor seria a adição de simulação aerodinâmica, no qual é possível saber o comportamento do chassis em alta velocidade, melhorando desempenho em tempo de volta e também consumo de combustível \cite{wang2018aerodynamic}. Outra possibilidade é uma implementação de um sistema de conselhos sobre o veículo, utilizando conceitos de inteligência artificial para obter melhores valores em ajustes do veículo. Este sistema vem em conjunto com a implementação de atuadores, onde os mesmo podem fazer alterações de parâmetros de peças como por exemplo a relação do câmbio em tempo de prova para melhor tração das rodas.   

Este trabalho foi de grande importância para o autor não apenas pelo conhecimento técnico adquiro. A participação no desenvolvimento de um projeto como o Baja SAE trouxe a necessidade de conciliar o que é importante para o projeto e o que os integrantes da equipe desejam fazer, criar um escopo de projeto com duração de um ano e a realização de um trabalho em uma equipe multidisciplinar com várias áreas da engenharia ajudou a expandir o entendimento sobre o que é trabalhar em colaboração de um objetivo maior e mais que isso, são conhecimentos que serão levados muito além deste trabalho.   