\chapter{Introdução} 
	\label{ch:introducao}

Veículos automotores estão presente no nosso dia a dia há mais de 100 anos \cite{fordt} e nunca param de evoluir. Tais  máquinas que por anos eram peças de engenharia mecânica tem passado por uma mudança de paradigma. Os carros tem acumulado eletrônica para desempenhar funções simples e complexas que algumas décadas atrás eram impossíveis. Sensores, atuadores e centrais eletrônicas de processamento podem ser usados para auxílio na segurança, com detecção de faixas de pista, auxílio em frenagem antecipada e controle de estabilidade \cite{racecarInstrumentationFor2012}, em aplicações de conforto com sensores de temperatura do ambiente para controlar aparelhos condicionadores de ar digital e também em aplicações de alto desempenho, onde os sistemas atuam para entregar um melhor tempo de volta para o veículo, de acordo com sua situação em que o mesmo se encontra na pista.

Ainda na área de alto desempenho a área de telemetria, aquisição de dados e tratamento de dados de um veículo são usadas de forma extensiva em competições automobilísticas onde a amostragem em tempo real é necessária para centenas de sensores que permitem aos engenheiros manterem o rastreamento do desempenho do \textit{powertrain}, parâmetros de dirigibilidade como configurações de suspensão e temperatura dos pneus \cite{designAndImplementation2015}. Estes dados são fundamentais para que os engenheiros e pilotos consigam acertar o carro a fim de obter tempo de voltas mais rápidos, além de melhor manutenção do carro de forma geral.

Dentro deste contexto, a equipe escolar de automobilismo Velociraptor\footnote{URL: \url{https://www.facebook.com/velociraptor.baja/}} foi procurada para realização de um projeto que envolvesse automobilismo e ciência da computação. Depois de diversas reuniões com a equipe foi encontrada uma área em comum, a aquisição de dados e telemetria do veículo da categoria Baja SAE codinome Fênix, que pode ser visto na Figura \ref{fig:Baja}. 


% O escopo trabalho desenvolvido teve diversas alterações durante sua realização. Inicialmente, o projeto englobaria telemetria, a criação de um sistema de controle \textit{on-board} (ou SCOB) e a criação de um \textit{software} de tratamento dos dados provenientes do veículo. Esta ideia possuía um escopo muito amplo para ser feito por apenas uma pessoa em dois semestres, então uma nova proposta foi realizada na qual o foco seria no \textit{software} de tratamento de dados recebendo estes mesmo dados via um \textit{Secure Digital Card}, ou como é normalmente conhecido cartão SD. Este segundo escopo era mais realista em relação ao tempo de execução do projeto, levando em conta o trabalho para a criação do sistema levantando os dados necessários para uma melhor forma de demostrar as grandezas do sistema.   
% A ideia inicial era a criação de um sistema completo de telemetria veicular para um veículo \textit{off-road}, desde a criação de um sistema de controle \textit{on-board} (ou SCOB) que consiste em uma unidade de processamento para dados recebidos de sensores espalhados pelo carro com um microcontrolador, até um \textit{software} de tratamento dos dados recebidos via telemetria. Esta ideia foi modificada com o tempo e visto que o escopo de trabalho seria muito abrangente e o período de dois semestres não seria suficiente.    

\begin{figure}[!htb]
	\centering
		\caption{Veículo Fênix da equipe Baja Velociraptor.}
		\includegraphics[scale=0.1]{baja} 
		\caption*{Fonte: Autor.}
		\label{fig:Baja}
\end{figure} 


O projeto consiste na criação de um \textit{software} de tratamento de dados e também na atualização e/ou criação de sistemas, sejam em veículo com outros \textit{softwares} ou \textit{hardware}, que auxiliariam este \textit{software}. O sistema de tratamento de dados deve receber informações de grandezas provenientes do veículo durante sua execução e mostrar em um computador nos boxes da equipe os dados mais importantes. Para realização deste sistema foi utilizado de telemetria com uma rede sem fio do protocolo ZigBee conectada ao sistema de controle \textit{on-board} (ou SCOB) que consiste em uma unidade de processamento para dados recebidos de sensores espalhados pelo carro com um microcontrolador.   

% Uma nova proposta então foi realizada, a criação de um \textit{software} de tratamento dos dados recebidos do SCOB por meio de um \textit{Secure Digital Card}, ou como é normalmente conhecido cartão SD. Contudo, durante reuniões do projeto do sub-sistema de eletrônica veicular do Velociraptor, foi adicionado ao escopo de projeto o uso de telemetria para recepção dos dados. Tendo esta oportunidade em mente uma nova proposta é então concebida para a segunda parte deste trabalho na qual todas as características anteriormente requisitadas se mantem, exceto pelo uso do cartão SD como meio principal de envio e recepção dos dados. 
%Nesta nova proposta, o escopo de projeto é mais consistente com a linha de aprendizado do curso de ciências da computação, mais realista em relação ao tempo de execução do projeto e mais coerente com o que a equipe Velociraptor espera deste projeto conjunto. 
% Contudo durante reuniões do projeto 2018 do sub-sistema de eletrônica veicular do Velociraptor, foi criado um terceiro escopo de projeto em conjunto com os interesses da equipe e suas prioridades. Neste escopo, que é o atual desenvolvido e testado neste trabalho, o sistema recebe as informações dos sensores via rede sem fio utilizando um par de módulos ZigBee. Este escopo se assemelha muito ao primeiro, porém com o auxilio da equipe de eletrônica para fabricação das placas de teste, a ideia enfim se tornou viável. 
Em relação ao sistema de tratamento de dados proposto, a equipe deseja obter melhores informações sobre o veículo de forma intuitiva com visualização dos dados com gráficos e separação das informações por setores automotivos, de forma simplificada com a utilização de um único programa para visualização das informações, com um programa feito de forma modular, ou seja, que esteja preparado para receber atualizações futuras de componentes e tecnologias e com armazenamento para melhor confiabilidade e comparações de informações de diversas provas de teste.   

Inicialmente alguns problemas encontrados no cenário de veículos \textit{off-road} podem ser analisados. Em \citeonline{projetoMiniBaja2006} alguns avisos sobre problemas que são comumente encontrados em situações de extremo esforço para o conjunto mecânico e eletrônico, como em uma prova de enduro do Baja SAE. Estes problemas devem ser levados em consideração na hora de montagem do sistema de aquisição de dados e também do \textit{software} de tratamento de dados. Os problemas encontrados são:

\begin{itemize}
	\item  Interferência eletro-magnética: os picos de tensão produzidos pela centelha da vela podem interferir no microcontrolador se o mesmo não for bem isolado; 
	\item Temperatura do \textit{cockpit}: depende muito da hora de realização do enduro e local, mas em regiões mais quentes como no nordeste e em horários entre 12 e 14 horas, a temperatura interna do \textit{cockpit} pode ficar muito elevada e esta pode comprometer qualquer circuito eletrônico que seja sensível a calor e não esteja protegido; 
	\item Vibração: pela natureza de uma prova \textit{off-road}, existe muita vibração. Este problema pode acarretar em mal contato entre circuitos e placas;
	\item Lama e água: devido ao contexto do projeto \textit{off-road} existem muitos obstáculos com lama e água no percurso dos testes e estes dois elementos podem danificar as placas eletrônicas.
\end{itemize}

Foram então levantados junto com a grupo Velociraptor várias grandezas físicas que a equipe deseja monitorar. Alguns componentes para sensoriamento já estão instalados no veículo e fazem parte dos requisitos do projeto, na parte de grandezas que devem ser monitoradas. Os seguintes itens foram levantados: 

\begin{itemize}
	\item Frequência de rotação do motor;
	\item Velocidade do veículo;
	\item Nível do combustível;
	\item Relação de transmissão;
	\item Temperatura do câmbio CVT;
	\item Rolagem da carroceria;
	\item Deslocamento do amortecedor;
	\item Deslocamento da suspensão; e
	\item Temperatura do disco de freio.
\end{itemize}

Enquanto algumas grandezas não fazem parte do escopo inicial deste projeto, as que fazem são analisadas na fundamentação teórica. As grandezas serão recebidas no computador dos boxes a partir do SCOB e transmitidas via protocolo ZigBee para a plataforma de tratamento de dados. Com este trabalho é esperado que a equipe consiga manter informações valiosas sobre o veículo que não poderiam ser obtidas de outra forma. A utilização de um \textit{software} de monitoramento é um passo importante para atingir um melhor desempenho na hora da manutenção, construção e realização de novos projetos para os veículos focados em automobilismo.   

\section{Estrutura do trabalho}

O trabalho a seguir é estruturado da seguinte forma. O Capítulo \ref{ch:fundamentacao} traz uma revisão de conceitos que serão utilizados no trabalho. Os conceitos abordados incluem o funcionamento e organização da prova da categoria Baja SAE, com alguns detalhes das regras impostas para produção do veículo \textit{off-road}, uma revisão sobre vários sensores utilizados para a captura de grandezas monitoradas, um estudo sobre microcontroladores para adequação de novas tecnologias sobre o sistema embarcado do veículo, uma análise sobre o protocolo de transmissão de dados sem fio e por fim um estudo sobre os objetivos dos sistemas de aquisição e tratamento de dados. No Capítulo \ref{ch:trabalhos} é realizado um estudo sobre o que existe de estado da arte em aquisição de dados na área automotiva, com foco em trabalhos voltados para o automobilismo. No Capítulo \ref{ch:proposta} é apresentada a proposta de trabalho, junto com os objetivos específicos que devem ser alcançados no percurso de sua realização. No Capítulo \ref{ch:desenvolvimento} é exposto o desenvolvimento do sistema de tratamento e de aquisição de dados. No Capítulo \ref{ch:testes} são discutidos os testes realizados no sistema para validação dos resultados. Por último no Capítulo \ref{ch:consideracoes} estão as conclusões sobre o trabalho e algumas ideias para trabalho futuros sobre o sistema especialmente para seu uso nas competições.  
