\chapter{\textit{Considerações Finais}}
	\label{ch:consideracoes}
O desenvolvimento de um \textit{software} de tratamento de dados para o veículo da categoria baja SAE se provou necessário devido ao nível de complexidade e quantidade de sensores que fazem e devem fazer parte do projeto da equipe Velociraptor. Apenas a utilização de arquivos de texto para gerenciamento das informações pode deixar a tarefa confusa e árdua e ainda sem garantias de resultados reais aonde a informação irá para a equipe. A utilização do programa de tratamento de dados foi validada tanto pela literatura revisada quanto por experiências tidas no curso da criação deste projeto.

Além da validação de conceito, a literatura também da algumas ideias relevantes para o projeto que devem ser analisadas para uma segunda parte. Em \citeonline{designAndImplementation2015} é comentado o uso de um \textit{Inertial Measurement Unit} com um sensor InvenSense MPU6050 que trabalha em conjunto com o SCOB para capturar informações de aceleração e velocidade angular. Em \citeonline{wirelessSensorNetwork2015}, mesmo que em um escopo diferente, é comentado o uso de um sensor MiniSense 100 para aquisição da grandeza de vibração, dado que pode ser útil para a prova de ergonomia realizada nos eventos do baja SAE.

Contudo, a literatura não trás muitas informações sobre quais os principais fatores devem ser levados em conta na hora da criação do \textit{software} de tratamento de dados. Devido ao fato de ser um cenário muito específico e a falta de colaboração entre equipes vista a natureza competitiva, estas informações serão levantadas junto a equipe na segunda parte deste trabalho. Após este levantamento, o principal objetivo será a produção deste sistema com a utilização de metodologias ágeis da engenharia de \textit{software} para garantir que os atributos desejáveis revisados neste trabalho sejam atingidos.        