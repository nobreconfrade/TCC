\chapter{Introdução} 
	\label{ch:introducao}

A área de telemetria, aquisição de dados e sensoriamento em geral de um veículo é usada de forma extensiva competições automobilisticas aonde amostragem em tempo real é necessária para centenas de sensores que permitem aos engenheiros manterem o rastreamento da performance do \textit{powertrain}, parametros de dirigibilidade como configurações de suspensão e temperatura dos pneus \cite{designAndImplementation2015}. Estes dados são fundamentais para que os engenheiros e pilotos consigam acertar o carro a fim de obter tempo de voltas mais rapidos, além de melhor manutenção do carro de forma geral.


Foram então levantados junto com o grupo Velociraptor os componentes que seriam monitorados. Os seguintes itens foram levantados: 
\begin{itemize}
	\item Frequência de rotação;
	\item Velocidade do veículo;
	\item Nível do combustível;
	\item Relação de transmissão;
	\item Temperatura do câmbio CVT;
	\item Rolagem da carroceria;
	\item Deslocamento do amortecedor;
	\item Deslocamento da suspensão;
	\item Temperatura do disco de freio;
\end{itemize}
\newpage

Os dados serão recebidos no computador de boxe a partir do \textit{SCOB} e transmitidos do cartão \textit{SD} interno para a plataforma para atualização dos dados.  

\section{Estrutura do trabalho}