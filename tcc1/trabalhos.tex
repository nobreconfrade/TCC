\chapter{\textit{Trabalhos Relacionados}}
	\label{ch:trabalhos}
Nesta seção será comentado sobre alguns trabalhos que possuem um objetivo similar ou utilizam tecnologias similares a este. 

O trabalho de \cite{gprsTelemetrySystem2013} propõe uma solução de telemetria para um veiculo de competição elétrico. A proposta é similar a do trabalho proposto no quesito de manter a equipe em questão atualizada dos dados provindos do carro, a diferença é que o veículo é movida a energia limpa, elétrica. Este requisito altera também a algumas das grandezas a serem analisadas pelo sistema, nele são verificados fatores como amperes por hora, voltagem, velocidade e distancia percorrida. Infelizmente o trabalho não comenta como é feita a coleta dos dados pelos sensores, apenas comentando que existe um equipamento que faz o mesmo. Para parte de telemetria, os desenvolvedores comentam que trabalharam com foco em resolver dois problemas: Robustez sobre todo o circuito de provas e segurança dos dados, além de redução do ruído. Portanto duas tecnologias foram analisadas para a transmissão dos dados embarcados do veículo até os computadores da equipe, a de radio frequência e a rede móvel de celular (GSM). A ultima foi escolhida devido a "(...) Impossibilidade de garantir a comunicação entre todos os circuitos devido aos formatos e obstáculos encontrados no terreno...", também é comentado que foram realizados testes e todos os circuitos possuem cobertura de sinal móvel. Por ultimo é desenvolvido um \textit{software} para receber os dados provenientes do veículo. O sistema de aquisição de dados teve seu funcionamento dividido em quatro blocos, sendo eles: Configuração do aplicativo e do canal transmissão de dados usado; Informações específicas da direção do piloto e do circuito percorrido; Valores numéricos dos dados técnicos mais importantes para a manutenção do veículo em tempo real; Representação gráfica de toda a informação recebida durante todo o processo da prova. Com isto e o tratamento das informações, o programa apresenta para a equipe de boxes informações como:

\begin{itemize}
	\item Consumo de energia; 
	\item Voltagem da bateria;
	\item Velocidade;
	\item Distância percorrida;
	\item Eficiência energética; 
\end{itemize}

Todas estas informações disponíveis são muito interessantes, porém \cite{gprsTelemetrySystem2013} não se aprofunda na sua construção, o que poderia aumentar em muito a relevância deste trabalho para o projeto proposto. Como é comentado, existe um indício de emprego de técnicas de engenharia de \textit{software} na criação do sistema de aquisição de dados, porém por não ser o foco do trabalho, as mesmas não são citadas. 

Outro trabalho que também possui um carro movido a energia limpa e propõe um sistema de aquisição de dados é \cite{applicationOfData2010}. O sistema é feito para um carro que utilizada um motor elétrico e deve ser capaz de percorrer distâncias de mais de 3000 quilômetros no evento \textit{World Solar Challenge}. Os equipamentos utilizados pela equipe foram o CompactRIO e LabVIEW da \textit{National Instrument}, para aquisição dos dados e a criação da plataforma de tratamento de dados, respectivamente, além do modulo de transmissão por radio frequência \textit{MaxStream} (atualmente \textit{Digi}) \textit{Xtream}. Como todos os dispositivos usados pela equipe são feitos por uma fabricantes externos, pouco é discutido sobre como funciona os sensores. O artigo demonstra um pouco sobre a arquitetura do sistema montado, utilizando os equipamentos citados e fala dos resultados, também não comentando muito sobre como foi feita a abordagem de criação do \textit{software} e que requisitos ele deve suprimir. Um dado interessante visto neste artigo é que para coletar dados de seis termopares, dois transdutores, um grupo de bateria e um tacômetro, foi necessário $363,3$ kilobytes de dados por hora, assim um cartão SD de dois gigabytes seria suficiente para armazenar uma longa bateria de treinos.

O artigo \cite{racecarInstrumentationFor2012} é mais abrangente, ele utiliza de um sistema de aquisição de dados e telemetria para estudar o comportamento de motoristas ao volante de carros convencionais. O estudo de comportamento visto no artigo não é verificado por não ser o foco, porém a parte de instrumentação faz algumas menções muito interessantes. Os testes feitos para as analises de dirigibilidade possuem alguns sensores construídos pelos autores, como o de posição do acelerador e sensores para cada roda a fim medir sua velocidade individual (útil em casos de derrapagem), e outra parte dos dados são pegos com um equipamento chamado \textit{Racelogic} VBOX, ele tira alguns dados como aceleração de 0 a 100 e distância percorrida utilizando GPS. Para as entradas analógicas foi comprado um sistema de aquisição de dados da \textit{National Instruments} modelo USB-6211 USB M Series (ou, como é chamado no artigo, NI-DAQ), onde tais entradas eram direcionadas a ele. Para os sensores de velocidade das rodas foi utilizado uma placa de prototipagem AVR-P40-USB-8535 da \textit{Olimex} em conjunto com um microcontrolador \textit{Microchip} da família ATMega32. O \textit{software} criado tem um codigo feito na linguagem C/C++, e o programa tem como objetivo se comunicar com a placa de prototipagem AVR e o sistema de aquisição de dados NI-DAQ, além de outros periféricos citados no texto do artigo. Tendo essas informações, o \textit{software} trata elas e disponibiliza para o usuário em tempo real, além de armazenar os dados em um arquivo de texto para análises posteriores. O \textit{software} tem taxa de atualização de 100 Hz, sendo a taxa de amostragem do NI-DAQ de 100Hz e a taxa de amostragem do sensores de roda de 20Hz no qual o valor final mostrado ao usuário é o ultimo dado recebido dos sensores. O artigo apresenta um diagrama, esquematizando o funcionamento do \textit{software}, e além da parte de sensoriamento ainda existem os algoritmos criados para o estudo de comportamento, aumentando o nível de complexidade total do sistema. Apenas o diagrama não seria suficiente para manter a equipe de desenvolvimento atualizada do progresso do \textit{software} proposto, mas como o foco do texto não esta na engenharia de \textit{software}, esta parte não se encontra discutida por completo.


Outro artigo revisado foi \cite{vehicleDataAcquisition2014}, este traz uma solução de aquisição de dados com um SCOB customizados para Formula SAE, porem o sistema é de uso geral na área veicular, podendo ser instalado em quaisquer outros modelos. O artigo não é muito específico em como e quais sensores são usados, ele coloca alguns exemplos como um sensor para temperatura (LM35) e como é feito um filtro matemático a partir da mediana de 100 valores para obter um resultado mais confiável. O foco do artigo é na parte de telemetria, nesta área é discutido uma solução para qual protocolo \textit{wireless} utilizar para o fim sensoriamento veicular. Algumas opções são descartadas no começo, como \textit{Bluetooth} e infravermelho, devido ao alcance limitado e a necessidade de manter contato direto entre os nós, o que é inviável em um circuito automobilístico. Então foram estudados dois outros protocolos, o WiFi e ZigBee. O artigo comenta que ambos possuem alcance mínimo para o cenário, ambos trabalham na frequência 2.4GHz e podem ter seus dados encriptados. Contudo o ZigBee foi escolhido devido a melhor relação de consumo de energia, além de, segundo o autor, ser mais simples de se instalar uma malha de rede ZigBee. A vantagem do WiFi é maiores taxas de transferência, porém no sistema analisado este requisito não era prioritário.         


Os artigos \cite{designAndImplementation2015} e \cite{developmentOfAn2016} são do mesmo autor no qual \cite{developmentOfAn2016} é desenvolvido um sistema de analise de comportamento na direção em cima de uma plataforma de aquisição de dados e \cite{designAndImplementation2015} é um artigo focado na construção da plataforma utilizada, sendo este o foco deste trabalho. Este sistema de aquisição de dados e monitoramento com telemetria foi especificamente feito para carros convencionais, pois ele utiliza uma entrada de leitura de dados padrão na maioria dos carros convencionais de rua. A \textit{On-Board Diagnostics} (Figura \ref{fig:obd}) é uma entrada presente geralmente na parte inferior do painel dos veículos e fornece informações diretamente da ECU do mesmo. Além de ser uma entrada padrão, os protocolos de interfaceamento para a retirada dos dados também são padronizados e independente de montadora, sendo está porta muito utilizada em oficinas para retirada de diagnósticos preliminares do veículo. Além desta fonte de informações, o autor também utiliza de um sensor de movimento MPU6050 para medir aceleração lateral e velocidade angular. Todos estes dados são enviados para um SCOB que diferente da maioria dos outros artigos aqui citados, consiste em um \textit{Raspberry Pi}. Isto é importante pois o artigo cita algumas vantagens interessantes de se ter uma \textit{PC Board} para processamento dos dados recebidos. Uma das vantagens é o poder de processamento superior em relação a um microcontrolador o que dependendo dos requisitos do projeto pode ser fundamental. Outra vantagem é a possibilidade de utilização de linguagens interpretadas como Python e Ruby, tudo graças ao ambiente que suporta um sistema operacional completo. Uma ultima vantagem comentada é a possibilidade de uso de um sistema gerenciador de bancos de dados integrados com o SCOB para \textit{backup} de informações e otimização do uso de memoria para armazenamento de dados, além de aumento na confiabilidade dos dados. Porém uma desvantagem que não é levantada mas deve ser levada em consideração é o aumento do custo monetário do projeto, com o preço de um \textit{Raspberry Pi} sendo aproximadamente dez vezes o preço de um \textit{Arduino Nano} (Novembro de 2017). O \textit{software} desenvolvido para este projeto utiliza \textit{Labview} e funciona via internet com uma interface web. Os dados são guardados dentro do SCOB e quando requisitado pelo computador são baixados, caso o computador não consiga fazer o carregamento das informações do SCOB, ele utiliza os dados presentes no servidor local. Existe um sistema de usuários para controle das informações e os dados dos sensores são divididos em categorias e podem ser visualizados em tabelas e gráficos. Nos testes, a taxa de amostragem de dados foi fixada em 100Hz e é comentado no artigo que o gargalo do sistema nesse quesito é o sistema \textit{On-Board Diagnostics}, que varia de carro para carro. Não é comentado nenhum tipo de planejamento ou analise de requisitos na parte do \textit{software} e também não é apresentado nenhum diagrama explicando o sistema corrente.    

\begin{figure}[!htb]
	\centering
		\includegraphics[height=6cm]{obd}
		\caption{Exemplo de entrada \textit{On-Board Diagnostics}. Fonte: Autor.}
		\label{fig:obd}
	\end{figure}

Foram encontrados alguns trabalhos com objetivo similar, como \cite{Dias2010} e \cite{Nunes2016} que tem propostas para criação de um sistema de telemetria para a modalidade baja SAE com um cenário muito similar ao denotado neste trabalho de conclusão de curso. \cite{Dias2010} Tem como foco a pesquisa, projeto e execução da parte de \textit{hardware} do sistema de telemetria, deixando a parte de \textit{software} para um segundo trabalho. Já \cite{Nunes2016} utiliza parte do que já foi projetado em outros anos na equipe Car-Kará para projetar um sistema completo de telemtria com duas ECUs, incluindo \textit{software} e \textit{hardware}.  

