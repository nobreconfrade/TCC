\resumo{Resumo}

%\textbf{\underline{PLACEHOLDER}}


\noindent Nas últimas décadas a indústria automotiva vem adicionando eletrônica para monitorar seus veículos em funcionamento urbano e para fins de performance em corridas automobilísticas. Neste trabalho é realizado um estudo de sensores, microcontroladores e processos de desenvolvimento de \textit{software} a fim de criar um sistema de aquisição e tratamento de dados para uma equipe de automobilismo escolar em uma categoria de veículos \textit{off-road}. A análise para produção do sistema abrange a aquisição das grandezas para uso em provas do evento baja SAE e também de medidas retiradas em testes, a escolha de um microcontrolador que melhor atende as necessidades futuras do sistema e como deve ser feita a produção deste \textit{software}, tendo como fundamentação a literatura encontrada em livros e artigos.  

%Eu gosto de carros. Desde criança minha paixão era correr nos circuitos virtuais de Gran Turismo, nas pistas noturnas de Need For Speed Underground e nas maluquíces de Top Gear 3000. Quando menor, sabia que para ser piloto era necessário muito dinheiro e minha família não possui tais recursos, então eu sonhava em ser um mecânico de carros, para poder mexer com os carros mais rápidos e os tornar-lós ainda mas rápidos, fazer com que a corrida não acabasse.\textbf{FAZER UM RESUMO DE VERDADE} 
\textbf{Palavras-chave:} \textit{Microcontroladores, Sensores, Engenharia de Software, Aquisição de Dados, baja SAE, Automobilismo}.

\resumo{Abstract}  
%\textbf{\underline{PLACEHOLDER}}


\noindent In the last decades the automotive industry have been acquiring eletronic devices for monitoring dynamic and comfort functions at urban areas and with performance focus on racetracks. This work portrays a study of sensors, microcontroller and software development methods with the objective of creating an acquisition and treatment system for data acquired from a off-road category car for a colege motorsport team. The analysis for the production of the system covers the acquisition of the quantities for use in the SAE competition, as well as measurements taken in tests. The choice of an microcontroller that best meets the future needs of the system and how the software should be produced is based on the literature found in books and articles.   


%I like cars. Since I was a kid my passion was to run in the virtual race tracks of Gran Turismo, in the night circuits of Need For Speed Underground and in the craziness of Top Gear 3000. When little, I knew it was needed a lot of money to be a motorsport race pilot and my family didn't had the resourcer to, so I dreamed to be a car mechanic, so I would be able to customize the fastest cars and make them go even faster, making sure the race never stop. \textbf{MAKE A REAL ABSTRACT}     
\noindent \textbf{Keywords:} \textit{Microcontrollers, Sensors, Software Engineering, Data Acquisition, baja SAE, Motorsport} 

\tableofcontents
\listoffigures
\listoftables
\newpage
\chapter*{Lista de Abreviaturas\hfill} \addcontentsline{toc}{chapter}{Lista de Abreviaturas}
\listofsymbols

\newpage
\pagestyle{myheadings}
